\documentclass{}

\begin{document}
EQUATION
 f(x) =x **3 -2*x - 5

BISECTION METHOD

Step 1: Choose the interval a b such that f(a) are opposite signs
	for ecample f(2) = -1 and  f(3) = 16
Step 2: Compute the midpont c 
		(a+b)/2 = c

		e.g. (2+3)/ 2 = 2.5
Step 3: Evaluate f(c)
	f(2.5) = 5.625
Sep 4: If f (c)= 0 , c is the root of f(x) and the algorithm terminates.
if not 
Step 5: if f(c) have opposite signs then the root is in the interval a c 
	set b= c
	
	e.g. our interval becomes 2 and 2.5
	if f(c) and f(b) have opposite signs then the root is in the interval b c so 
	set a= c

	
Step 6: Repeat step 2

	




if maximum number of iteration is reached terminate 



NEWTON RAMPASION METHOD

Step 1: Choose an initial guess Xa
	Xa = 0

Step 2: Evaluate the function f(x) and its derivative f'(x) at Xa
		If Xa = 0
	          f(0) = -5
		f'(0) = -2
sTEP 3: Compute the next approximation X(n +1)
	Xa - (f(0) /f'(0)) = 0 - (-5/-2) = -2.5
Step 4: if it is less than the tolerance then that is the root
		it is not less that the root
	if it is not then the new Xa is the computed approximation 
		-2.5 becomes our new Xa
Step 5: Go back to step 2
	if Xa = -2.5
	f(0) =5.625
	f'(0) = 16.75
	 

	X(n+1) = -2.5 - (5.625/1.75) = -2.8358
	go back to step 2

the function convergrs when the answer is cllose to the root




\end{document}